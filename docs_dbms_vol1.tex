% DBMS Volume 1 - Basics used in the Healthcare Mini DBMS project
\documentclass[11pt,a4paper]{article}
\usepackage[utf8]{inputenc}
\usepackage{geometry}
\usepackage{hyperref}
\usepackage{listings}
\usepackage{color}
\usepackage{longtable}
\usepackage{footnote}
\usepackage{titlesec}
\geometry{margin=1in}

\definecolor{codebg}{rgb}{0.95,0.95,0.95}
\lstset{backgroundcolor=\color{codebg},basicstyle=\ttfamily\small,breaklines=true}

\title{DBMS Volume 1: Basic Database Concepts used in the Healthcare Mini DBMS}
\author{Project: Healthcare Mini DBMS\\Volume 1: Core DBMS Concepts}
\date{\today}

\begin{document}
\maketitle
\tableofcontents
\thispagestyle{empty}
\newpage

\section*{Preface}
This first volume covers the basic database (DBMS) concepts used in the Healthcare Mini DBMS project. Each chapter gives a short theory section, then shows the exact syntax used in the project (with file locations), and finally explains how the project implements or uses that feature.

File locations referenced below are relative to the project root (e.g. \texttt{schema.sql}, \texttt{app.py}, \texttt{cli.py}, \texttt{init_db.py}, \texttt{migrate_db.py}, \texttt{seed_data.py}).

\section{Relational Model and Schema}
\subsection{Theory}
The relational model represents data as tables (relations) with rows (tuples) and columns (attributes). A schema defines tables, their columns and types, and constraints (keys, foreign keys, uniqueness).

Key concepts:
\begin{itemize}
  \item Table (relation): collection of rows with same columns.
  \item Primary key (PK): unique identifier for rows.
  \item Foreign key (FK): column that references a PK in another table.
  \item Data types: TEXT, INTEGER, REAL, DATETIME, DATE in SQLite.
\end{itemize}

\subsection{Exact syntax used (schema)}
The project's schema is in \texttt{schema.sql}. Below is a representative example (exact file: \texttt{schema.sql}):
\begin{lstlisting}
CREATE TABLE patients (
    patient_id INTEGER PRIMARY KEY AUTOINCREMENT,
    first_name TEXT NOT NULL,
    last_name TEXT NOT NULL,
    dob DATE,
    gender TEXT,
    phone TEXT,
    email TEXT,
    address TEXT,
    insurance TEXT
);
\end{lstlisting}

Another example showing a foreign key and ON DELETE behavior (also in \texttt{schema.sql}):
\begin{lstlisting}
CREATE TABLE visits (
    visit_id INTEGER PRIMARY KEY AUTOINCREMENT,
    appointment_id INTEGER,
    patient_id INTEGER NOT NULL,
    doctor_id INTEGER NOT NULL,
    visit_date DATETIME DEFAULT CURRENT_TIMESTAMP,
    diagnosis TEXT,
    notes TEXT,
    FOREIGN KEY (appointment_id) REFERENCES appointments(appointment_id) ON DELETE SET NULL,
    FOREIGN KEY (patient_id) REFERENCES patients(patient_id) ON DELETE CASCADE,
    FOREIGN KEY (doctor_id) REFERENCES doctors(doctor_id) ON DELETE CASCADE
);
\end{lstlisting}

\subsection{How the project uses the schema}
\begin{itemize}
  \item \texttt{init_db.py} applies the schema to create the database file (\texttt{hospital.db}). See the code in \texttt{init_db.py} where it reads and executes \texttt{schema.sql}.
  \item Tables mirror real-world entities: \texttt{patients}, \texttt{doctors}, \texttt{appointments}, \texttt{visits}, \texttt{prescriptions}, \texttt{bills}.
  \item Foreign keys with ON DELETE CASCADE / SET NULL are used so deletions in parent tables cascade as appropriate — e.g., deleting a patient removes visits and bills.
\end{itemize}

\section{Data Definition Language (DDL)}
\subsection{Theory}
D D L commands create and modify schema objects: \texttt{CREATE TABLE}, \texttt{ALTER TABLE}, \texttt{DROP TABLE}, \texttt{CREATE INDEX}.

\subsection{Exact syntax used}
Examples are in \texttt{schema.sql}. See table creation snippets above. The project also uses a trigger creation statement in \texttt{migrate_db.py}:
\begin{lstlisting}
CREATE TRIGGER IF NOT EXISTS trg_create_bill_after_visit
AFTER INSERT ON visits
BEGIN
    INSERT INTO bills (visit_id, patient_id, amount, status)
    VALUES (NEW.visit_id, NEW.patient_id, 50.0, 'unpaid');
END;
\end{lstlisting}

\subsection{How the project uses DDL}
\begin{itemize}
  \item \texttt{init_db.py} executes the DDL (the entire \texttt{schema.sql}) to create the tables and constraints.
  \item \texttt{migrate_db.py} applies a trigger (DDL) to implement automatic billing after a visit is inserted.
\end{itemize}

\section{Data Manipulation Language (DML): INSERT / SELECT / UPDATE / DELETE}
\subsection{Theory}
DML commands change or query data: \texttt{INSERT}, \texttt{SELECT}, \texttt{UPDATE}, \texttt{DELETE}. Parameterized queries prevent SQL injection in applications.

\subsection{Exact syntax examples from the code}
Insertion (used in \texttt{init_db.py}, \texttt{seed\_data.py}, and application):
\begin{lstlisting}
cur.execute('INSERT INTO patients (first_name, last_name, dob, phone, email, address, insurance) VALUES (?, ?, ?, ?, ?, ?, ?)',
            (first, last, dob, phone, email, address, insurance))
conn.commit()
\end{lstlisting}

Select with joins (from \texttt{app.py} appointments view):
\begin{lstlisting}
cur.execute('''
    SELECT a.appointment_id, a.appointment_datetime, a.status, a.reason,
           p.patient_id, p.first_name AS patient_first, p.last_name AS patient_last,
           d.doctor_id, d.first_name AS doctor_first, d.last_name AS doctor_last
    FROM appointments a
    JOIN patients p ON a.patient_id = p.patient_id
    JOIN doctors d ON a.doctor_id = d.doctor_id
    ORDER BY a.appointment_datetime
''')
rows = cur.fetchall()
\end{lstlisting}

Update example (mark a bill paid; used in scripts or could be used in app):
\begin{lstlisting}
cur.execute('UPDATE bills SET status = ?, paid_at = ? WHERE bill_id = ?', ('paid', paid_at_str, bill_id))
conn.commit()
\end{lstlisting}

Delete example (used by the new web delete route in \texttt{app.py}):
\begin{lstlisting}
cur.execute('DELETE FROM patients WHERE patient_id = ?', (patient_id,))
conn.commit()
\end{lstlisting}

\subsection{How the project uses DML}
\begin{itemize}
  \item The Flask app and CLI perform DML for CRUD operations: add patients, schedule appointments, create visits, insert prescriptions and bills.
  \item Parameterized queries with \texttt{?} placeholders are used throughout (see \texttt{app.py} and \texttt{cli.py}), which prevents SQL injection and is best practice.
  \item The trigger auto-inserts bills on visit insertion (an example of DML executed by DDL-trigger logic).
\end{itemize}

\section{Constraints: Primary keys, Foreign keys, UNIQUE, NOT NULL}
\subsection{Theory}
Constraints keep data integrity: \texttt{PRIMARY KEY}, \texttt{FOREIGN KEY}, \texttt{UNIQUE}, \texttt{NOT NULL}, \texttt{CHECK}.

\subsection{Exact syntax examples}
From \texttt{schema.sql}:
\begin{lstlisting}
patient_id INTEGER PRIMARY KEY AUTOINCREMENT,
first_name TEXT NOT NULL,
email TEXT UNIQUE,
FOREIGN KEY (patient_id) REFERENCES patients(patient_id) ON DELETE CASCADE,
\end{lstlisting}

\subsection{How the project uses constraints}
\begin{itemize}
  \item Primary keys: every table defines an integer PK (e.g., \texttt{patient_id}, \texttt{doctor_id}).
  \item NOT NULL used where the data must exist (e.g., patient first/last name, appointment_datetime).
  \item FOREIGN KEYS enforce relationships and use ON DELETE actions (CASCADE or SET NULL) to maintain referential integrity automatically.
  \item UNIQUE on doctor email prevents duplicate contact entries.
\end{itemize}

\section{Transactions and Atomicity}
\subsection{Theory}
Transactions ensure groups of statements succeed or fail together (atomicity). SQLite autocommit mode can be controlled; using \texttt{conn.commit()} writes changes; \texttt{conn.rollback()} can undo.

\subsection{Exact code references}
Code uses explicit commits after DML (see \texttt{init_db.py}, \texttt{app.py}):
\begin{lstlisting}
conn.commit()
\end{lstlisting}

\subsection{How the project uses transactions}
\begin{itemize}
  \item Database write operations call \texttt{conn.commit()} after successful inserts/updates.
  \item The sample scripts do not explicitly wrap multiple statements in manual BEGIN/COMMIT blocks because SQLite Python driver will manage transactions around executes unless you call \texttt{isolation_level=None}. For critical multi-step changes you can use \texttt{conn.execute('BEGIN')} and explicit rollback handling.
\end{itemize}

\section{Triggers}
\subsection{Theory}
Triggers are database-side procedures that run automatically in response to DML events (INSERT/UPDATE/DELETE). They are useful for enforcing invariants or maintaining derived data.

\subsection{Exact syntax used}
See \texttt{migrate_db.py} for the trigger definition (quoted earlier). The trigger inserts a bill row after each new visit.

\subsection{How the project uses triggers}
\begin{itemize}
  \item Ensures every visit that is created has an associated bill inserted automatically (business rule enforced at DB layer).
  \item This keeps application code simpler: app inserts a visit and the DB creates billing rows.
\end{itemize}

\section{Queries, Joins and Aggregation}
\subsection{Theory}
Joins combine rows from multiple tables. Aggregation functions (SUM, COUNT) summarize data. GROUP BY groups result sets to aggregate per key.

\subsection{Exact syntax examples from project}
Billing report in \texttt{app.py}:
\begin{lstlisting}
SELECT p.patient_id, p.first_name || ' ' || p.last_name AS patient_name,
       SUM(b.amount) AS total_billed,
       SUM(CASE WHEN b.status = 'unpaid' THEN b.amount ELSE 0 END) AS total_unpaid
FROM bills b
JOIN patients p ON b.patient_id = p.patient_id
GROUP BY p.patient_id, patient_name
ORDER BY total_unpaid DESC
\end{lstlisting}

Doctor workload report (LEFT JOIN with date filter):
\begin{lstlisting}
SELECT d.doctor_id, d.first_name || ' ' || d.last_name AS doctor_name,
       COUNT(a.appointment_id) AS upcoming_appointments
FROM doctors d
LEFT JOIN appointments a ON a.doctor_id = d.doctor_id
    AND date(a.appointment_datetime) BETWEEN date('now') AND date('now', '+7 days')
GROUP BY d.doctor_id, doctor_name
ORDER BY upcoming_appointments DESC
\end{lstlisting}

\subsection{How the project uses queries}
\begin{itemize}
  \item Reports use aggregations and joins to produce dashboards (billing, doctor workload, daily appointments, overdue bills).
  \item Application list views use JOINs to display related names (appointment page joins patients/doctors).
\end{itemize}

\section{Normalization and Design Decisions}
\subsection{Theory}
Normalization organizes schema to reduce redundancy (1NF, 2NF, 3NF). Sometimes controlled denormalization is used for auditability or performance.

\subsection{Project choices}
\begin{itemize}
  \item The project models medications in a separate \texttt{medications} table, but also stores free-text \texttt{medication} on the \texttt{prescriptions} table. This gives both a canonical reference (\texttt{med_id}) and a historical record (free-text), which is a deliberate denormalization for auditability.
  \item Many-to-one relationships (appointments → patient, doctor) are normalized.
\end{itemize}

\section{Referential Actions (ON DELETE)}
\subsection{Theory}
Foreign keys can specify actions like CASCADE, SET NULL, RESTRICT on delete/update to control what happens to child rows.

\subsection{Examples in project}
\begin{lstlisting}
FOREIGN KEY (patient_id) REFERENCES patients(patient_id) ON DELETE CASCADE,
FOREIGN KEY (department_id) REFERENCES departments(department_id) ON DELETE SET NULL
\end{lstlisting}

\subsection{How the project uses them}
\begin{itemize}
  \item Deleting a patient cascades to appointments, visits and bills so the DB remains consistent.
  \item Deleting a department sets department references to NULL on doctors/appointments (department is optional).
\end{itemize}

\section{Practical tips and best practices used}
\begin{itemize}
  \item Use parameterized queries (\texttt{?} placeholders) to avoid injection.
  \item Keep business-critical invariants in the DB where possible (triggers, constraints).
  \item Use explicit commits after writes to make transaction boundaries clear.
  \item Keep seed/test data separate from schema creation (this project now initializes schema-only by default in \texttt{init_db.py}).
\end{itemize}

\section*{Appendix: File map (where to look in code)}
\begin{longtable}{p{0.3\textwidth} p{0.65\textwidth}}
\texttt{schema.sql} & Full DDL for tables and constraints.\
\texttt{init_db.py} & Applies schema to create \texttt{hospital.db} (now schema-only).\
\texttt{migrate_db.py} & Migration that creates the billing trigger.\
\texttt{app.py} & Flask app: examples of DML (INSERT/SELECT/UPDATE/DELETE) and report queries.\
\texttt{cli.py} & Click-based CLI commands invoking SQL for manual/automated tasks.\
\texttt{seed_data.py} & (optional) programmatic seeder to populate mock data for demos/tests.\
\end{longtable}

\section*{Wrap up}
This volume covered the essential DBMS concepts employed in the project and tied each concept to exact code locations/syntax used. If you'd like, I will produce Volume 2 covering intermediate topics (indexes, query tuning, concurrency, backups, testing, migrations) or expand each chapter with hands-on exercises and step-by-step walkthroughs.

\end{document}
